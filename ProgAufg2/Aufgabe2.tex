\documentclass[11pt,titlepage]{article}

%Laenderspezifische Einstellungen bzgl. Rechtschreibung, Sonderzeichen und Kodierung
\usepackage[utf8]{inputenc}
\usepackage[naustrian]{babel}
\usepackage[T1]{fontenc}
\usepackage{titlesec}
\usepackage{graphicx}
%\usepackage{subcaption}

\usepackage{listings}
\usepackage{color}
\usepackage{courier}
\usepackage{matlab-prettifier}
\definecolor{light-gray}{gray}{0.85}
\lstset{
language=C++,
numbers=left,
style=Matlab-editor,
basicstyle=\mlttfamily,
breaklines=true,
backgroundcolor=\color{light-gray},
tabsize=2,
basicstyle=\footnotesize\ttfamily,
frame=single,
inputencoding=utf8,
extendedchars=true,
showstringspaces=false,
literate =
	{ä}{{\"a}}1
	{ö}{{\"o}}1
	{ü}{{\"u}}1
	{Ä}{{\"A}}1
	{Ö}{{\"O}}1
	{Ü}{{\"U}}1
	{ß}{{\ss}}1
	{ₙ}{{$_n$}}1
}

\def\ContinueLineNumber{\lstset{firstnumber=last}}
\def\StartLineAt#1{\lstset{firstnumber=#1}}

\usepackage[
	a4paper,
	top = 2cm,
	bottom = 2 cm,
	left = 2cm,
	right = 2cm,
	headheight = 15pt,
	includeheadfoot
	]{geometry}
\usepackage{fancyhdr}
\usepackage{amssymb}
\usepackage{amsmath}
\usepackage[english]{varioref}
\usepackage{hyperref}

\fancypagestyle{fancy}{
	\fancyhead[R]{Page \thepage}
	\fancyhead[L]{\leftmark}
	\renewcommand{\headrulewidth}{1.25pt}

	\fancyfoot[L]{\tiny{Num. Ana. - Übung 2 , created: \today}}
	\fancyfoot[R]{\tiny{Felix Dreßler (k12105003), Elisabeth Köberle (k12110408), Ricardo Holzapfel (k11942080)}}
	\cfoot{}
	\renewcommand{\footrulewidth}{1.25pt}
}

\setlength{\headsep}{10mm}
\setlength{\footskip}{10mm}

\setlength{\parindent}{0mm}
\setlength{\parskip}{1.1ex plus0.25ex minus0.25ex}
\setlength{\tabcolsep}{0.2cm} % for the horizontal padding

\pagestyle{fancy}

\title{Num. Ana. - Übung 2}
\author{Felix Dreßler (k12105003) \\ Elisabeth Köberle (k12110408) \\ Ricardo Holzapfel (k11942080)}
\date{\today} %Erstellungsdatum

\begin{document}
\maketitle

	\section{Testfunktion 1}
		\begin{displaymath}
			\int_{-1.25}^1 \sqrt{|x|} \, dx\
		\end{displaymath}
		Zuerst berechnen wir mithilfe der Matlab-Funktion \glqq integral\grqq \, das numerische Ergebnis des Integrals.
		\begin{lstlisting}
			>> fun = @(x) sqrt(abs(x))
			
			fun =
			
			function_handle with value:
			
			@(x)sqrt(abs(x))
			
			>> integral(fun, -1.25,1)
			
			ans =
			
			1.598362906722469
		\end{lstlisting}
		Nun testen wir unsere Funktion mit Standardwerten.
		\begin{lstlisting}
			>> [I, exitflag] = E_Trapez(@testfun1, -1.25, 1)
			
			I =
			
			1.598399277022473
			
			
			exitflag =
			
			1
		\end{lstlisting}
		Mit $m_{max} = 6$ konvergiert das Verfahren noch nicht zufriedenstellen, dazu benötigen wir $m_{max} = 24$:
		\begin{lstlisting}
			>> [I, exitflag] = E_Trapez(@testfun1, -1.25, 1, 24)
			
			I =
			
			1.598361657434845
			
			
			exitflag =
			
			0
		\end{lstlisting}
		Die Lösung unterscheidet sich jedoch noch in der sechsten Nachkommastelle von der Matlab Lösung.
		
	\section{Testfunktion 2}
		\begin{displaymath}
			\int_0^{\pi} sin(x) \, dx\
		\end{displaymath}
		Zuerst berechnen wir mithilfe der Matlab-Funktion \glqq integral\grqq \, das numerische Ergebnis des Integrals.
		\begin{lstlisting}
			>> fun = @(x) sin(x)
			
			fun =
			
			function_handle with value:
			
			@(x)sin(x)
			
			>> integral(fun, 0,pi)
			
			ans =
			
			2.000000000000000
		\end{lstlisting}
		Nun testen wir unsere Funktion mit Standardwerten.
		\begin{lstlisting}
			>> [I, exitflag] = E_Trapez(@testfun2, 0, pi)
			
			I =
			
			2.000000000000008
			
			
			exitflag =
			
			0
		\end{lstlisting}
		Hier erhalten wir eine Abweichung an der 15-ten Nachkommastelle zum Matlab Ergebnis.
	
	\section{Testfunktion 3}
		\begin{displaymath}
			\int_{1.1}^{3.7} e^{x} - \dfrac{x^2}{2}  \, dx\
		\end{displaymath}
		Zuerst berechnen wir mithilfe der Matlab-Funktion \glqq integral\grqq \, das numerische Ergebnis des Integrals.
		\begin{lstlisting}
			>> fun = @(x) exp(x)-(x.^2)/2
			
			fun =
			
			function_handle with value:
			
			@(x)exp(x)-(x.^2)/2
			
			>> integral(fun, 1.1,3.7)
			
			ans =
			
			29.222805002787631
		\end{lstlisting}
		Nun testen wir unsere Funktion mit Standardwerten.
		\begin{lstlisting}
			>> [I, exitflag] = E_Trapez(@testfun3, 1.1, 3.7)
			
			I =
			
			29.212897596611022
			
			
			exitflag =
			
			1
		\end{lstlisting}
		Um das Abbruchskriterium des Verfahrens zu erreichen musste hier entweder $m_{max}$ sehr groß gewählt oder die Genauigkeit verringert werden.
		Mit verringerter Genauigkeit ist das Ergebnis jedoch wie hier ersichtlich schon ungenauer, als bei obiger Ausführung.
		\begin{lstlisting}
			>> [I, exitflag] = E_Trapez(@testfun3, 1.1, 3.7, 6, ceil((3.7-1.1)/0.1), 1.e-3)
			
			I =
			
			29.182340731811546
			
			
			exitflag =
			
			0
		\end{lstlisting}
		
	\section{Testfunktion 4}
		Da wir vor allem bei \glqq einfachen\grqq \, Funktionen genaue Ergebnisse gewünscht sind, testen wir auch solche:
		\begin{displaymath}
			\int_{0}^1 x \, dx\
		\end{displaymath}
		Zuerst berechnen wir mithilfe der Matlab-Funktion \glqq integral\grqq \, das numerische Ergebnis des Integrals.
		\begin{lstlisting}
			>> fun = @(x) x
			
			fun =
			
			function_handle with value:
			
			@(x)x
			
			>> integral(fun, 0, 1)
			
			ans =
			
			0.500000000000000
		\end{lstlisting}
		Auch mit $m_{max} = 20$ bricht das Programm noch mit $exitflag = 1$ ab.
		\begin{lstlisting}
			>> [I, exitflag] = E_Trapez(@testfun4, 0, 1, 20)
			
			I =
			
			0.499999980635652
			
			
			exitflag =
			
			1
		\end{lstlisting}
		Deshalb verringern wir hier auch die Genauigkeit.
		\begin{lstlisting}
			>> [I, exitflag] = E_Trapez(@testfun4, 0, 1, 6, ceil((1)/0.1), 1.e-3)
			
			I =
			
			0.499351851851852
			
			
			exitflag =
			
			0
		\end{lstlisting}

	\section{Testfunktion 5}
		Schnell wachsende Integrale sind ebenfalls Interessant zu betrachten:
		\begin{displaymath}
			\int_{0}^1 2^x \, dx\
		\end{displaymath}
		Zuerst berechnen wir mithilfe der Matlab-Funktion \glqq integral\grqq \, das numerische Ergebnis des Integrals.
		\begin{lstlisting}
			>> fun = @(x) 2.^x
			
			fun =
			
			function_handle with value:
			
			@(x)2.^x
			
			>> integral(fun, 0, 1)
			
			ans =
			
			1.442695040888963
		\end{lstlisting}
		Nun testen wir unsere Funktion mit Standardwerten.
		\begin{lstlisting}
			>> [I, exitflag] = E_Trapez(@testfun5, 0, 1)
			
			I =
			
			1.442377527965380
			
			
			exitflag =
			
			1
		\end{lstlisting}
		Das Programm bricht mit $exitflag = 1$ ab, deshalb verringern wir die Genauigkeit.
		\begin{lstlisting}
			>> [I, exitflag] = E_Trapez(@testfun5, 0, 1, 6, ceil((1)/0.1), 1.e-3)
			
			I =
			
			1.441398244528734
			
			
			exitflag =
			
			0
		\end{lstlisting}

	\section{Testfunktion 6}
		Als letzte Testfunktion wurde folgende Funktion gewählt:
		\begin{displaymath}
			\int_{-1}^1  x^2 \, dx\
		\end{displaymath}
		Zuerst berechnen wir mithilfe der Matlab-Funktion \glqq integral\grqq \, das numerische Ergebnis des Integrals.
		\begin{lstlisting}
			>> fun = @(x) x.^2
			
			fun =
			
			function_handle with value:
			
			@(x)x.^2
			
			>> integral(fun, -1, 1)
			
			ans =
			
			0.666666666666667
		\end{lstlisting}
		Nun testen wir unsere Funktion mit Standardwerten.
		\begin{lstlisting}
			>> [I, exitflag] = E_Trapez(@testfun6, -1, 1)
			
			I =
			
			0.666666541732919
			
			
			exitflag =
			
			1
		\end{lstlisting}
		Da das Programm wieder mit $exitflag = 1$ abbricht, deshalb erhöhen wir $m_{max}$
		\begin{lstlisting}
			>> [I, exitflag] = E_Trapez(@testfun6, -1, 1, 11)
			
			I =
			
			0.666666666662302
			
			
			exitflag =
			
			0
			
		\end{lstlisting}
		Mit $m_{max} = 11$ bricht das Programm erfolgreich ab.
		
	\section{Resultate}
		 Wenn das Programm mit den Standardwerten erfolgreich abbricht, erhalten wir duchaus genau Ergebnise in relativ kurzer Zeit. Erhöhen wir jedoch $m_{max}$ kann das Verfahren schnell einiges an Systemressourcen benötigen.
		 
		 Beispielsweise könnten wir bei der dritten Testfunktion anstatt der Verringerung der Genauiggkeit auch $m_{max}$ erhöhen. Das führt jedoch schnell zu einem erheblichen RAM-Verbrauch. 
		 
		 Versuchen wir nämlich die Funktion mit $m_{max} = 50$ auszuführen, werden schnell bis zu 30 GB RAM verwendet, der Vorgang dauert ebenso sehr lange. (Vermutung - Matlab greift auf viel langsamere \glqq swap-memory\grqq \, zu)
		 	 
		 Vermutlich resultiert dieser Verbrauch daraus, dass wir zuerst für jedes $m$ das ganze Tableau speichern müssen. Das könnte vermutlich durch eine etwas abgeänderte Abbruchsvorschrift gelöst werden.
		 
		 Das ganze Tableau zu speichern hat natürlich den Vorteil, dass dadurch pro Iterationsschritt über m die bereits berechneten Werte nicht neu berechnet werden müssen.
		 
		 Ebenfalls ergibt sich ein Speicher-Overhead durch das Speichern der linken unteren Dreiecksmatrix T, welche noch viele unnötig gespeicherte Nullen enthält.
		
			
\end{document}