\documentclass[11pt,titlepage]{article}

%Laenderspezifische Einstellungen bzgl. Rechtschreibung, Sonderzeichen und Kodierung
\usepackage[utf8]{inputenc}
\usepackage[naustrian]{babel}
\usepackage[T1]{fontenc}
\usepackage{titlesec}
\usepackage{graphicx}
%\usepackage{subcaption}

\usepackage{listings}
\usepackage{color}
\usepackage{courier}
\usepackage{matlab-prettifier}
\definecolor{light-gray}{gray}{0.85}
\lstset{
language=C++,
numbers=left,
style=Matlab-editor,
basicstyle=\mlttfamily,
breaklines=true,
backgroundcolor=\color{light-gray},
tabsize=2,
basicstyle=\footnotesize\ttfamily,
frame=single,
inputencoding=utf8,
extendedchars=true,
showstringspaces=false,
literate =
	{ä}{{\"a}}1
	{ö}{{\"o}}1
	{ü}{{\"u}}1
	{Ä}{{\"A}}1
	{Ö}{{\"O}}1
	{Ü}{{\"U}}1
	{ß}{{\ss}}1
	{ₙ}{{$_n$}}1
}

\def\ContinueLineNumber{\lstset{firstnumber=last}}
\def\StartLineAt#1{\lstset{firstnumber=#1}}

\usepackage[
	a4paper,
	top = 2cm,
	bottom = 2 cm,
	left = 2cm,
	right = 2cm,
	headheight = 15pt,
	includeheadfoot
	]{geometry}
\usepackage{fancyhdr}
\usepackage{amssymb}
\usepackage{amsmath}
\usepackage[english]{varioref}
\usepackage{hyperref}

\fancypagestyle{fancy}{
	\fancyhead[R]{Page \thepage}
	\fancyhead[L]{\leftmark}
	\renewcommand{\headrulewidth}{1.25pt}

	\fancyfoot[L]{\tiny{Num. Ana. - Übung 2 , created: \today}}
	\fancyfoot[R]{\tiny{Felix Dreßler (k12105003), Elisabeth Köberle (k12110408), Ricardo Holzapfel (k11942080)}}
	\cfoot{}
	\renewcommand{\footrulewidth}{1.25pt}
}

\setlength{\headsep}{10mm}
\setlength{\footskip}{10mm}

\setlength{\parindent}{0mm}
\setlength{\parskip}{1.1ex plus0.25ex minus0.25ex}
\setlength{\tabcolsep}{0.2cm} % for the horizontal padding

\pagestyle{fancy}

\title{Num. Ana. - Übung 2}
\author{Felix Dreßler (k12105003) \\ Elisabeth Köberle (k12110408) \\ Ricardo Holzapfel (k11942080)}
\date{\today} %Erstellungsdatum

\begin{document}
\maketitle

	\section{Testfunktion 1}
		\begin{displaymath}
			\int_{-1.25}^1 \sqrt{|x|} \, dx\
		\end{displaymath}
		Zuerst berechnen wir mithilfe der Matlab-Funktion \glqq integral\grqq das numerische Ergebnis des Integrals.
		\begin{lstlisting}
			>> fun = @(x) sqrt(abs(x))
			
			fun =
			
			function_handle with value:
			
			@(x)sqrt(abs(x))
			
			>> integral(fun, -1.25,1)
			
			ans =
			
			1.598362906722469
		\end{lstlisting}
		Nun testen wir unsere Funktion mit Standardwerten.
		\begin{lstlisting}
			
		\end{lstlisting}
		
		
	\section{Testfunktion 2}
		\begin{displaymath}
			\int_0^{\pi} sin(x) \, dx\
		\end{displaymath}
		Zuerst berechnen wir mithilfe der Matlab-Funktion \glqq integral\grqq das numerische Ergebnis des Integrals.
		\begin{lstlisting}
			>> fun = @(x) sin(x)
			
			fun =
			
			function_handle with value:
			
			@(x)sin(x)
			
			>> integral(fun, 0,pi)
			
			ans =
			
			2.000000000000000
		\end{lstlisting}

	\section{Testfunktion 3}
		\begin{displaymath}
			\int_{1.1}^{3.7} e^{x} - \dfrac{x^2}{2}  \, dx\
		\end{displaymath}
		Zuerst berechnen wir mithilfe der Matlab-Funktion \glqq integral\grqq das numerische Ergebnis des Integrals.
		\begin{lstlisting}
			>> fun = @(x) exp(x)-(x.^2)/2
			
			fun =
			
			function_handle with value:
			
			@(x)exp(x)-(x.^2)/2
			
			>> integral(fun, 1.1,3.7)
			
			ans =
			
			29.222805002787631
		\end{lstlisting}

	\section{Testfunktion 4}
		Da wir vor allem bei \glqq einfachen\grqq Funktionen genaue Ergebnisse gewünscht sind, testen wir auch solche:
		\begin{displaymath}
			\int_{0}^1 x \, dx\
		\end{displaymath}
		Zuerst berechnen wir mithilfe der Matlab-Funktion \glqq integral\grqq das numerische Ergebnis des Integrals.
		\begin{lstlisting}
			>> fun = @(x) x
			
			fun =
			
			function_handle with value:
			
			@(x)x
			
			>> integral(fun, 0, 1)
			
			ans =
			
			0.500000000000000
		\end{lstlisting}

	\section{Testfunktion 5}
		Schnell wachsende Integrale sind ebenfalls Interessant zu betrachten:
		\begin{displaymath}
			\int_{0}^7 2^x \, dx\
		\end{displaymath}
		Zuerst berechnen wir mithilfe der Matlab-Funktion \glqq integral\grqq das numerische Ergebnis des Integrals.
		\begin{lstlisting}
			>> fun = @(x) 2.^x
			
			fun =
			
			function_handle with value:
			
			@(x)2.^x
			
			>> integral(fun, 0, 7)
			
			ans =
			
			1.832222701928983e+02
		\end{lstlisting}

	\section{Testfunktion 6}
		Als letzte Testfunktion wurde folgende Funktion gewählt:
		\begin{displaymath}
			\int_{-1}^1  x^2 \, dx\
		\end{displaymath}
		Zuerst berechnen wir mithilfe der Matlab-Funktion \glqq integral\grqq das numerische Ergebnis des Integrals.
		\begin{lstlisting}
			>> fun = @(x) x.^2
			
			fun =
			
			function_handle with value:
			
			@(x)x.^2
			
			>> integral(fun, -1, 1)
			
			ans =
			
			0.666666666666667
		\end{lstlisting}
			
\end{document}